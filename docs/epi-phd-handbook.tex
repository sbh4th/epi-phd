% Options for packages loaded elsewhere
\PassOptionsToPackage{unicode}{hyperref}
\PassOptionsToPackage{hyphens}{url}
%
\documentclass[
]{book}
\usepackage{lmodern}
\usepackage{amsmath}
\usepackage{ifxetex,ifluatex}
\ifnum 0\ifxetex 1\fi\ifluatex 1\fi=0 % if pdftex
  \usepackage[T1]{fontenc}
  \usepackage[utf8]{inputenc}
  \usepackage{textcomp} % provide euro and other symbols
  \usepackage{amssymb}
\else % if luatex or xetex
  \usepackage{unicode-math}
  \defaultfontfeatures{Scale=MatchLowercase}
  \defaultfontfeatures[\rmfamily]{Ligatures=TeX,Scale=1}
\fi
% Use upquote if available, for straight quotes in verbatim environments
\IfFileExists{upquote.sty}{\usepackage{upquote}}{}
\IfFileExists{microtype.sty}{% use microtype if available
  \usepackage[]{microtype}
  \UseMicrotypeSet[protrusion]{basicmath} % disable protrusion for tt fonts
}{}
\makeatletter
\@ifundefined{KOMAClassName}{% if non-KOMA class
  \IfFileExists{parskip.sty}{%
    \usepackage{parskip}
  }{% else
    \setlength{\parindent}{0pt}
    \setlength{\parskip}{6pt plus 2pt minus 1pt}}
}{% if KOMA class
  \KOMAoptions{parskip=half}}
\makeatother
\usepackage{xcolor}
\IfFileExists{xurl.sty}{\usepackage{xurl}}{} % add URL line breaks if available
\IfFileExists{bookmark.sty}{\usepackage{bookmark}}{\usepackage{hyperref}}
\hypersetup{
  pdftitle={   PhD Handbook },
  pdfauthor={PhD Program Leadership},
  hidelinks,
  pdfcreator={LaTeX via pandoc}}
\urlstyle{same} % disable monospaced font for URLs
\usepackage{longtable,booktabs}
\usepackage{calc} % for calculating minipage widths
% Correct order of tables after \paragraph or \subparagraph
\usepackage{etoolbox}
\makeatletter
\patchcmd\longtable{\par}{\if@noskipsec\mbox{}\fi\par}{}{}
\makeatother
% Allow footnotes in longtable head/foot
\IfFileExists{footnotehyper.sty}{\usepackage{footnotehyper}}{\usepackage{footnote}}
\makesavenoteenv{longtable}
\usepackage{graphicx}
\makeatletter
\def\maxwidth{\ifdim\Gin@nat@width>\linewidth\linewidth\else\Gin@nat@width\fi}
\def\maxheight{\ifdim\Gin@nat@height>\textheight\textheight\else\Gin@nat@height\fi}
\makeatother
% Scale images if necessary, so that they will not overflow the page
% margins by default, and it is still possible to overwrite the defaults
% using explicit options in \includegraphics[width, height, ...]{}
\setkeys{Gin}{width=\maxwidth,height=\maxheight,keepaspectratio}
% Set default figure placement to htbp
\makeatletter
\def\fps@figure{htbp}
\makeatother
\setlength{\emergencystretch}{3em} % prevent overfull lines
\providecommand{\tightlist}{%
  \setlength{\itemsep}{0pt}\setlength{\parskip}{0pt}}
\setcounter{secnumdepth}{5}
\usepackage{booktabs}
\usepackage{booktabs}
\usepackage{longtable}
\usepackage{array}
\usepackage{multirow}
\usepackage{wrapfig}
\usepackage{float}
\usepackage{colortbl}
\usepackage{pdflscape}
\usepackage{tabu}
\usepackage{threeparttable}
\usepackage{threeparttablex}
\usepackage[normalem]{ulem}
\usepackage{makecell}
\usepackage{xcolor}
\ifluatex
  \usepackage{selnolig}  % disable illegal ligatures
\fi
\usepackage[]{natbib}
\bibliographystyle{apalike}

\title{\includegraphics[width=3.125in,height=\textheight]{mcgill-epi-logo.png} PhD Handbook}
\author{PhD Program Leadership}
\date{2021-08-05}

\begin{document}
\maketitle

{
\setcounter{tocdepth}{1}
\tableofcontents
}
\hypertarget{preface}{%
\chapter*{Preface}\label{preface}}
\addcontentsline{toc}{chapter}{Preface}

Note that this handbook is specific to the Epidemiology program and does not replace McGill's Graduate and Postdoctoral Studies (GPS) \href{https://www.mcgill.ca/students/courses/calendars/}{Calendar} or \href{https://www.mcgill.ca/gps/students/policies-and-guidelines}{Policies and Procedures}. You are responsible for reading and understanding the official GPS procedures, rules and regulations. Please contact the Epidemiology Graduate Program Director (GPD) or Student Services Office to answer any questions.

\hypertarget{section}{%
\section{---}\label{section}}

\hypertarget{introduction}{%
\chapter{Introduction}\label{introduction}}

This provides an overview of the PhD program in Epidemiology at McGill.

Epidemiology is the study and analysis of the patterns and causes of disease in human populations. It forms the core discipline of public health by identifying the distribution and determinants of health and disease, and by gaining the etiologic understanding to intervene toward the improvement of population health. The PhD program in epidemiology at McGill trains scientists and health professionals to design and conduct studies, analyze health data and effectively communicate scientific results, and to gain novel insights into the causes and prevention of diseases at the population level. Epidemiologic work at the doctoral level involves a thorough integration of biological knowledge of pathogenesis, statistical knowledge of quantitative analysis and causal inference, and sociological knowledge to place these insights in the context of dynamic and interconnected human populations. Major areas of strength at McGill include epidemiologic methods, clinical epidemiology, infectious diseases, social epidemiology, pharmacoepidemiology, public and population health, global health, environmental epidemiology, chronic diseases and aging, and perinatal epidemiology.

\hypertarget{competencies}{%
\section{Competencies}\label{competencies}}

Our program aims to prepare our students for successful careers in epidemiology. Upon successful completion of the PhD in Epidemiology at McGill, we aim for our students to:

\begin{itemize}
\tightlist
\item
  Understand the difference between descriptive and etiologic epidemiologic studies, and the value of both designs for public health science;
\item
  Develop a thorough understanding of modern epidemiologic methods and how they are utilized in the service of descriptive, predictive, and etiologic study designs;
\item
  Apply quantitative skills to the analysis of epidemiologic data using statistical software;
\item
  Systematically and critically review the epidemiologic literature, synthesize existing evidence, and identify important gaps in knowledge;
\item
  Design, write, and critique an independent research proposal for answering epidemiologic questions;
\end{itemize}

\hypertarget{high-level-program-overview}{%
\section{High-Level Program Overview}\label{high-level-program-overview}}

Successful completion of the PhD program in EBOH involves 4 key milestones:\\
- Required coursework;\\
- Passing a comprehensive exam;\\
- Developing and defending a thesis protocol; and\\
- Writing and defending the doctoral thesis.

The timeline for program completion varies depending on each student's circumstances and subject matter, but most of our students complete the PhD in around 5 years.

\hypertarget{section-1}{%
\section{---}\label{section-1}}

\hypertarget{coursework}{%
\chapter{Coursework}\label{coursework}}

PhD students are required to complete 25 course credits, including 16 required credits and 9 elective credits.

\hypertarget{required-courses}{%
\section{Required courses}\label{required-courses}}

The required coursework is typically completed during the first 4 terms and consists of the following courses:

\begin{itemize}
\tightlist
\item
  EPIB 701 PhD Comprehensive Examination*
\item
  EPIB 702 PhD Proposal*
\item
  EPIB 703 Principles of Study Design (2 Credits)
\item
  EPIB 704 Doctoral Level Epidemiologic Methods 1 (4 Credits)
\item
  EPIB 705 Doctoral Level Epidemiologic Methods 2 (4 Credits)
\item
  EPIB 706 Doctoral Seminar in Epidemiology (3 Credits)
\item
  EPIB 707 Research Design in the Health Sciences (3 Credits)
\end{itemize}

*Note that EPIB 701 and 702 are not didactic courses but are required milestones for advancing toward degree completion and require registration in the appropriate term. Students register for EPIB 702 in both Fall and Winter terms of their second year.

\hypertarget{elective-courses}{%
\section{Elective courses}\label{elective-courses}}

Students are also required to take 9 credits of elective coursework, at the 500 level or higher, with a minimum of 3 credits in Biostatistics and 6 credits in an epidemiologic and/or substantive topic (normally related to the thesis topic). Elective courses must be chosen in consultation with the student's supervisor and/or the degree program's director or adviser.

These courses can be chosen from the Department's current offer of more than 40 courses in EBOH as well as from other McGill Departments. To assist you in your course selections see the Ph.D.~Epidemiology Electives Guidelines page. Below in Table \ref{tab:elect} you can find a list of current EBOH courses commonly taken as electives. However, courses from other departments or faculties may be possible, depending on the thesis subject matter and subject to the approval of your supervisor(s) and the Program Director.

\begin{table}

\caption{\label{tab:elect}EBOH Electives as of Fall 2021}
\centering
\resizebox{\linewidth}{!}{
\begin{tabular}[t]{l|r|l}
\hline
Course & Credits & Elective Category\\
\hline
\cellcolor{gray!6}{BIOS 612 Advanced Generalized Linear Models} & \cellcolor{gray!6}{4} & \cellcolor{gray!6}{Biostatistics}\\
\hline
BIOS 624 Data Analysis \& Report Writing & 4 & Biostatistics\\
\hline
\cellcolor{gray!6}{BIOS 691 Bayesian Analysis in the Health Sciences 1} & \cellcolor{gray!6}{3} & \cellcolor{gray!6}{Biostatistics}\\
\hline
EPIB 625 Ethics of Human Research & 3 & Epi/Substantive\\
\hline
\cellcolor{gray!6}{EPIB 627 Analysis of Correlated Data} & \cellcolor{gray!6}{3} & \cellcolor{gray!6}{Biostatistics}\\
\hline
EPIB 628 Measurement in Epidemiology & 3 & Epi/Substantive\\
\hline
\cellcolor{gray!6}{EPIB 629 Knowledge Synthesis} & \cellcolor{gray!6}{3} & \cellcolor{gray!6}{Epi/Substantive}\\
\hline
EPIB 631 Pharmacoepidemiology 2 & 2 & Epi/Substantive\\
\hline
\cellcolor{gray!6}{EPIB 632 Mental Disorders: Population Perspectives and Methods} & \cellcolor{gray!6}{3} & \cellcolor{gray!6}{Epi/Substantive}\\
\hline
EPIB 633 Pharmacoepidemiology 1 & 2 & Epi/Substantive\\
\hline
\cellcolor{gray!6}{EPIB 635 Clinical Trials} & \cellcolor{gray!6}{3} & \cellcolor{gray!6}{Epi/Substantive}\\
\hline
EPIB 637 Advanced Survival Analysis & 3 & Biostatistics\\
\hline
\cellcolor{gray!6}{EPIB 638 Mathematical Modeling of Infectious Diseases} & \cellcolor{gray!6}{3} & \cellcolor{gray!6}{Epi/Substantive}\\
\hline
EPIB 639 Pharmacoepidemiology Methods & 4 & Epi/Substantive\\
\hline
\cellcolor{gray!6}{EPIB 645 Confounding Control in Pharmacoepidemiology} & \cellcolor{gray!6}{1} & \cellcolor{gray!6}{Epi/Substantive}\\
\hline
EPIB 647 Analysis of Temporal and Spatial Data & 3 & Epi/Substantive\\
\hline
\cellcolor{gray!6}{EPIB 648 Methods in Social Epidemiology} & \cellcolor{gray!6}{3} & \cellcolor{gray!6}{Epi/Substantive}\\
\hline
EPIB 654 Pharmacoepidemiology 4 & 2 & Epi/Substantive\\
\hline
\cellcolor{gray!6}{EPIB 661 Pharmacoepidemiology 3} & \cellcolor{gray!6}{2} & \cellcolor{gray!6}{Epi/Substantive}\\
\hline
EPIB 662 Pharmacological Basis of Pharmacoepidemiology & 1 & Epi/Substantive\\
\hline
\cellcolor{gray!6}{EPIB 671 Cancer Epidemiology\&Prevention} & \cellcolor{gray!6}{2} & \cellcolor{gray!6}{Epi/Substantive}\\
\hline
EPIB 675 Special Topics: Health Care Systems Anaylsis Using Administrative Data & 3 & Epi/Substantive\\
\hline
\cellcolor{gray!6}{EPIB 676 Special Topics: Bayesian Analysis in the Health Sciences} & \cellcolor{gray!6}{3} & \cellcolor{gray!6}{Biostatistics}\\
\hline
EPIB 679 Special Topics: Genetic Epidemiology & 3 & Epi/Substantive\\
\hline
\cellcolor{gray!6}{EPIB 681 Global Health: Epid. Research} & \cellcolor{gray!6}{3} & \cellcolor{gray!6}{Epi/Substantive}\\
\hline
EPIB 684 Princ of Envrnmntl Hlth Sci 1 & 3 & Epi/Substantive\\
\hline
\cellcolor{gray!6}{EPIB 685 Princ of Envrnmntl Hlth Sci 2} & \cellcolor{gray!6}{3} & \cellcolor{gray!6}{Epi/Substantive}\\
\hline
EPIB 686 Environmental Health Seminar & 3 & Epi/Substantive\\
\hline
\cellcolor{gray!6}{PPHS 501 Population Health and Epidemiology} & \cellcolor{gray!6}{3} & \cellcolor{gray!6}{Epi/Substantive}\\
\hline
PPHS 511 Fundamentals of Global Health & 3 & Epi/Substantive\\
\hline
\cellcolor{gray!6}{PPHS 525 Healthcare Systems in Comparative Perspective} & \cellcolor{gray!6}{3} & \cellcolor{gray!6}{Epi/Substantive}\\
\hline
PPHS 527 Economics for Health Services Research and Policy & 3 & Epi/Substantive\\
\hline
\cellcolor{gray!6}{PPHS 528 Economic Evaluation of Health Programs} & \cellcolor{gray!6}{3} & \cellcolor{gray!6}{Epi/Substantive}\\
\hline
PPHS 529 Global Environmental Health and Burden of Disease & 3 & Epi/Substantive\\
\hline
\cellcolor{gray!6}{PPHS 612 Principles/Pub Hlth Practice} & \cellcolor{gray!6}{3} & \cellcolor{gray!6}{Epi/Substantive}\\
\hline
PPHS 613 The Practice of Global Health & 3 & Epi/Substantive\\
\hline
\cellcolor{gray!6}{PPHS 614 Knowledge Translation and Public Health Leadership} & \cellcolor{gray!6}{3} & \cellcolor{gray!6}{Epi/Substantive}\\
\hline
PPHS 615 Intro:Infectious Disease Epid & 3 & Epi/Substantive\\
\hline
\cellcolor{gray!6}{PPHS 616 Principles \& Practice of Public Health Surveillance} & \cellcolor{gray!6}{3} & \cellcolor{gray!6}{Epi/Substantive}\\
\hline
PPHS 617 Impact Evaluation & 3 & Epi/Substantive\\
\hline
\cellcolor{gray!6}{PPHS 618 Program Planning and Evaluation in Public Health} & \cellcolor{gray!6}{3} & \cellcolor{gray!6}{Epi/Substantive}\\
\hline
PPHS 624 Public Health Ethics \& Policy & 3 & Epi/Substantive\\
\hline
\cellcolor{gray!6}{PPHS 682 Special Topics: Critical Perspectives on Global Health} & \cellcolor{gray!6}{2} & \cellcolor{gray!6}{Epi/Substantive}\\
\hline
PPHS 683 Special Topics: Vaccine Epidemiology & 3 & Epi/Substantive\\
\hline
\cellcolor{gray!6}{PPHS 684 Special Topics: Foundations of Health Promotion} & \cellcolor{gray!6}{3} & \cellcolor{gray!6}{Epi/Substantive}\\
\hline
\end{tabular}}
\end{table}

\hypertarget{directed-reading-courses}{%
\section{Directed Reading Courses}\label{directed-reading-courses}}

\emph{Directed Reading courses complement offerings in the department or elsewhere at McGill or other universities. They are NOT substitutes for existing courses but are, rather, ways for students in the programs to enrich their education in an organized way on topics not otherwise covered or not covered sufficiently (in depth or breadth) in existing courses.}

Students enrolled in the department may take Directed Reading courses for credit towards a degree under the rubric of the Special Topics offerings. These courses may be for 1, 2, or 3 credits. Directed Reading courses should conform to the usual semester format unless the specific circumstances of the course require flexibility. However, students are expected to complete such a course within no more than any six month period. Students will be expected to submit for approval in advance material that provides the objectives and methods to be used for the directed reading work.

There is considerable flexibility in what constitutes a directed reading course, but certain requirements must be met before work can begin, including:

\begin{enumerate}
\def\labelenumi{\arabic{enumi}.}
\item
  Students must themselves propose a supervising faculty member with whom to work.
\item
  With the faculty supervisor, students must prepare an adequate project proposal commensurate with the number of credits sought that includes:
\end{enumerate}

\begin{itemize}
\item
  The rationale for doing this work as a Directed Reading course and for the number of credits sought. As well, this statement should indicate how it relates to, but is separate from, thesis work when the student is in a thesis program.
\item
  An outline of the work to be done and the final product/output to be submitted. If a Reading Course is being proposed, a preliminary bibliography and a planned reading schedule should be included.
\item
  A timetable, with appropriate milestones to assess a student's progress and the measures to be used to evaluate the work (e.g., number of written assignments and their length). A student's faculty supervisor will be responsible for this evaluation as is the case for ``regular'' courses.
\item
  A timetable indicating when the student and faculty supervisor will meet.
\end{itemize}

\begin{enumerate}
\def\labelenumi{\arabic{enumi}.}
\setcounter{enumi}{2}
\tightlist
\item
  The project proposal, signed by both the student and the supervisor, should be submitted to the Student Affairs Office a minimum of one month prior to the start of the semester in which the course will take place. The director, along with one other person on the Program Committee who has accepted responsibility for curriculum matters, will review the proposal and determine if it is to be approved. Once approved internally, a copy will be sent to the Director of Graduate Studies as well as to the Department's Graduate Studies Office, with a request that the latter obtain a Special Topics course number for the offering. A copy of the final approved version of the course content will be placed in the student's file.
\end{enumerate}

\hypertarget{example-curriculum}{%
\section{Example curriculum}\label{example-curriculum}}

The timing and choices to fulfill the course requirements will likely be unique for each student. Below we provide one example of a possible curriculum over the course of the program.

\begin{itemize}
\tightlist
\item
  Year 1

  \begin{itemize}
  \tightlist
  \item
    Fall: EPIB 703 Study Design; EPIB 704 Epi Methods I; Ethics Requirement: Tri-Council Policy for Ethical Conduct of Research online module (TCPS-2) (non-credit)\\
  \item
    Winter: EPIB 705 Epi Methods II; EPIB 706 Doctoral Seminar
  \item
    Summer: EPIB 701 Comprehensive Exam (June)
  \end{itemize}
\item
  Year 2

  \begin{itemize}
  \tightlist
  \item
    Fall: EPIB 702 PhD proposal; EPIB 707 Research Design; BIOS elective (e.g., Advanced Generalized Linear Models, Causal Inference)
  \item
    Winter: EPIB 702 PhD proposal; EPIB or substantive electives (e.g.~Pharmacoepidemiology, Impact Evaluation, Knowledge Synthesis)
  \end{itemize}
\item
  Year 3

  \begin{itemize}
  \tightlist
  \item
    Fall:
  \end{itemize}
\end{itemize}

\hypertarget{section-2}{%
\section{---}\label{section-2}}

\hypertarget{concentrations}{%
\chapter{Concentrations}\label{concentrations}}

There are presently 3 options for Epidemiology PhD students that wish to pursue concentrated work in substantive areas related to either Global Health, Pharmacoepidemiology, or Population Dynamics. In addition to the other milestones for the PhD degree (Comprehensive Exam, Protocol Defence, Thesis Defence), these options all require additional courses to be completed \emph{in addition to} the required courses for all Epidemiology PhD students.

\hypertarget{global-health-option}{%
\section{Global Health Option}\label{global-health-option}}

This option will provide enhanced training in global health to graduate students registered in the PhD in Epidemiology; Global Health degree program at McGill. Students will become familiar with topics of global health relevance and incorporate this into their core coursework and thesis research. The thesis must be relevant to global health and approved by the Global Health Coordinating Committee. Contextualizing the core training students receive in epidemiology and in their respective substantive discipline within the global health research domain will enhance their academic experience. Graduates of this option will be prepared to pursue further training in global health or to undertake a variety of career opportunities in global health in Canada or internationally. The PhD thesis must be relevant to global health and approved by the Global Health Coordinating Committee.

\textbf{Required Courses (22 credits)}\\
\emph{Option-specific courses are italicized}

\begin{itemize}
\tightlist
\item
  \emph{EPIB 681 Global Health: Epidemiologic Research (3 Credits)}
\item
  \emph{PPHS 511 Fundamentals of Global Health (3 Credits)}
\item
  EPIB 701 PhD Comprehensive Examination
\item
  EPIB 702 PhD Proposal
\item
  EPIB 703 Principles of Study Design (2 Credits)
\item
  EPIB 704 Doctoral Level Epidemiologic Methods 1 (4 Credits)
\item
  EPIB 705 Doctoral Level Epidemiologic Methods 2 (4 Credits)
\item
  EPIB 706 Doctoral Seminar in Epidemiology (3 Credits)
\item
  EPIB 707 Research Design in the Health Sciences (3 Credits)
\end{itemize}

\textbf{Complementary Courses (9 credits)}

6 credits of coursework at the 500 level or higher, with a minimum of 3 credits in Biostatistics, and 3 credits in Epidemiology. Courses must be chosen in consultation with the student's supervisor and/or the degree program's director or adviser.

3 credits of coursework at the 500 level or higher from this list, or any other course \emph{approved by the Global Health Option Committee} that have not been taken to satisfy other program requirements.

\begin{itemize}
\tightlist
\item
  GEOG 503 Advanced Topics in Health Geog 3 Credits
\item
  NUTR 501 Nutrition in Dev Countries 3 Credits
\item
  PPHS 525 HlthCare Systems in Comp Persp 3 Credits
\item
  PPHS 527 Econ for Hlth Serv Res\&Policy 3 Credits
\item
  PPHS 529 Global Env Hlth\&Burden/Disease 3 Credits
\item
  SOCI 513 Soc Aspects HIV/AIDS in Africa 3 Credits
\item
  SOCI 519 Gender and Globalization 3 Credits
\item
  SOCI 545 Sociology of Population 3 Credits
\end{itemize}

Please contact the Global Health Option Advisor \href{mailto:madhukar.pai@mcgill.ca}{Madhu Pai} for any questions regarding this option.

\hypertarget{pharmacoepidemiology-option}{%
\section{Pharmacoepidemiology Option}\label{pharmacoepidemiology-option}}

This program provides in-depth training for graduate students on pharmacoepidemiologic methods and the application of these methods to study the population effects (benefits and harms) of pharmaceutical products. Students will acquire the skills to become independent investigators and conduct original research in pharmacoepidemiology. Career opportunities for graduates are multiple and include work in industry, government, or academia. Students will be required to participate in the Pharmacoepidemiology Journal Club. Research topics must be related to pharmacoepidemiology and approved by the program coordinating committee.

\textbf{Required Courses (25 credits)}\\
\emph{Option-specific courses are italicized}

\begin{itemize}
\tightlist
\item
  \emph{EPIB 639 Pharmacoepidemiologic Methods (4 Credits)}
\item
  \emph{EPIB 654 Pharmacoepidemiology 4 (2 Credits)}
\item
  \emph{EPIB 661 Pharmacoepidemiology 3 (2 Credits)}
\item
  \emph{EPIB 662 Pharma Basis of Pharmacoepidem (1 Credit)}
\item
  EPIB 701 PhD Comprehensive Examination
\item
  EPIB 702 PhD Proposal
\item
  EPIB 703 Principles of Study Design (2 Credits)
\item
  EPIB 704 Doctoral Level Epidemiologic Methods 1 (4 Credits)
\item
  EPIB 705 Doctoral Level Epidemiologic Methods 2 (4 Credits)
\item
  EPIB 706 Doctoral Seminar in Epidemiology (3 Credits)
\item
  EPIB 707 Research Design in the Health Sciences (3 Credits)
\end{itemize}

\textbf{Complementary Courses (3 credits)}

3 credits of coursework in Biostatistics at the 500 level or higher. Courses must be chosen in consultation with the student's supervisor and/or the degree program's director or adviser.

These courses can be chosen from the Department's current offer of more than 40 courses in Epidemiology, Biostatistics and Occupational Health as well as from other McGill Departments.

Please contact the Global Health Option Advisor \href{mailto:robert.platt@mcgill.ca}{Robert Platt} for any questions regarding this option.

\hypertarget{population-dynamics-option}{%
\section{Population Dynamics Option}\label{population-dynamics-option}}

The Population Dynamics Option (PDO) is a cross-disciplinary, cross-faculty graduate program offered by the Centre on Population Dynamics (CPD) as an option within existing master's and doctoral degree programs in the Departments of Sociology, Economics, and Epidemiology, Biostatistics and Occupational Health (EBOH) at McGill. Students who have been admitted through their home department or faculty may apply for admission to the option. The option is coordinated by the CPD, in partnership with participating academic units.

Thus, in addition to the rigorous training provided in the Department of EBOH, graduate students who choose this option become Centre on Population Dynamics (CPD) student trainees. This affiliation notably offers opportunities for interdisciplinary research and supervision. The option also provides a forum whereby graduate students bring their disciplinary perspectives together and enrich each other's learning through structured courses, a weekly seminar series, and informal discussions and networking.

With interdisciplinary research being increasingly important to understanding complex social and biological processes, CPD student trainees benefit from both a strong disciplinary foundation from their departmental affiliations, as well as from the sharing of knowledge across disciplinary boundaries through CPD activities.

\textbf{Required Courses (22 credits)}\\
\emph{Option-specific courses are italicized}

\begin{itemize}
\tightlist
\item
  EPIB 701 PhD Comprehensive Examination
\item
  EPIB 702 PhD Proposal
\item
  EPIB 703 Principles of Study Design (2 Credits)
\item
  EPIB 704 Doctoral Level Epidemiologic Methods 1 (4 Credits)
\item
  EPIB 705 Doctoral Level Epidemiologic Methods 2 (4 Credits)
\item
  EPIB 706 Doctoral Seminar in Epidemiology (3 Credits)
\item
  EPIB 707 Research Design in the Health Sciences (3 Credits)
\item
  \emph{SOCI 545 Sociology of Population (3 Credits)}
\item
  \emph{SOCI 626 Demographic Methods (3 Credits)}
\end{itemize}

\textbf{Complementary Courses (9 credits)}

9 credits of coursework, at the 500 level or higher, with a minimum of 3 credits in Biostatistics, 3 credits in Epidemiology, and 3 credits from courses approved for the Population Dynamics Option from the list below:

\begin{itemize}
\tightlist
\item
  ECON 622 Public Finance (3 Credits)
\item
  ECON 634 Economic Development 3 (3 Credits)
\item
  ECON 641 Labour Economics (3 Credits)
\item
  ECON 734 Economic Development 4 (3 Credits)
\item
  ECON 741 Advanced Labour Economics (3 Credits)
\item
  ECON 742 Empirical Microeconomics (3 Credits)
\item
  ECON 744 Health Economics (3 Credits)
\item
  EPIB 648 Methods in Social Epidemiology (3 Credits)
\item
  EPIB 681 Global Health: Epidemiological Research (3 Credits)
\item
  PPHS 525 Health Care Systems in Comparative Perspective (3 Credits)
\item
  PPHS 528 Economic Eval of Hlth Programs (3 Credits)
\item
  PPHS 529 Global Env Hlth\&Burden/Disease (3 Credits)
\item
  PPHS 615 Intro:Infectious Disease Epid (3 Credits)
\item
  SOCI 502 Sociology of Fertility (3 Credits)
\item
  SOCI 512 Ethnicity \& Public Policy (3 Credits)
\item
  SOCI 513 Soc Aspects HIV/AIDS in Africa (3 Credits)
\item
  SOCI 520 Migration and Immigrant Groups (3 Credits)
\item
  SOCI 535 Sociology of the Family (3 Credits)
\item
  SOCI 588 Biosociology/Biodemography (3 Credits)
\end{itemize}

Courses must be chosen in consultation with the student's supervisor and/or the degree program's director or adviser.

Please contact the Global Health Option Advisor \href{mailto:amelie.quesnelvallee@mcgill.ca}{Amelie Quesnel-Valee} for any questions regarding this option.

\hypertarget{section-3}{%
\section{---}\label{section-3}}

\hypertarget{comprehensive-exam}{%
\chapter{Comprehensive Exam}\label{comprehensive-exam}}

Doctoral students will normally take the Comprehensive Exam (EPIB 701) within 12 to 24 months of entry into the Ph.D.~degree program. The Comprehensive Exam is held once a year in June.

\hypertarget{purpose-needs-revising-and-expanding}{%
\section{\texorpdfstring{Purpose (\emph{needs revising and expanding})}{Purpose (needs revising and expanding)}}\label{purpose-needs-revising-and-expanding}}

The exam is intended to test students' ability to synthesize and integrate epidemiological knowledge.

\hypertarget{content-and-structure-needs-revising-and-expanding}{%
\section{\texorpdfstring{Content and Structure (\emph{needs revising and expanding})}{Content and Structure (needs revising and expanding)}}\label{content-and-structure-needs-revising-and-expanding}}

For details on the Comprehensive Exam, see the course outline for EPIB 701

\hypertarget{evaluation-needs-revising-and-expanding}{%
\section{\texorpdfstring{Evaluation (\emph{needs revising and expanding})}{Evaluation (needs revising and expanding)}}\label{evaluation-needs-revising-and-expanding}}

The exam is graded ``Pass'' or ``Fail''. Students with a ``Fail'' must repeat the exam the following year.

\hypertarget{reporting}{%
\section{Reporting}\label{reporting}}

Results of the comprehensive exam will be transmitted to students within 3(2?) weeks of the date of the exam.

\hypertarget{section-4}{%
\section{---}\label{section-4}}

\hypertarget{thesis-protocol}{%
\chapter{Thesis Protocol}\label{thesis-protocol}}

\hypertarget{purpose}{%
\section{Purpose}\label{purpose}}

All PhD students will prepare and defend a protocol for their thesis project. This is typically after students have passed the Comprehensive Exam after the first year of require coursework, though some students may elect to register for the course before passing their comprehensive examination, while defending their own protocol afterwards. It is intended to provide the student with the experience necessary to propose a comprehensive research project and to convince their peers of its scientific merit and originality. It provides a standardized critical evaluation
to supplement the expertise of the student's PhD supervisor and committee and thereby improve the likelihood of the thesis' success.

\hypertarget{format}{%
\section{Format}\label{format}}

The written protocol should emphasize the importance of the research objective(s) and the proposed methods for addressing them. As per McGill University policy, it can be written in English or French. It should provide sufficient background and detail on data sources; research design, statistical analyses, and power/precision/sample size (as applicable) for each of the research objectives; and study limitations. The student should make clear in the protocol text the extent of their original contribution to the proposed research, as well as the likely format (``traditional'' vs manuscript-based) of the thesis.

\hypertarget{assessment}{%
\section{Assessment}\label{assessment}}

{[}\emph{text likely needs revision}{]}
The thesis protocol is reviewed by two faculty members plus an external reviewer with substantive expertise in the student's specific area of research. The student, thesis supervisor(s), and other thesis committee members are responsible for seeking the participation of the external reviewer. Importantly, the external reviewer must not be a member of the thesis committee, have previously supervised or co-authored a paper with the student, nor have been consulted about any aspect of the proposed research.

\hypertarget{section-5}{%
\section{---}\label{section-5}}

\hypertarget{phd-thesis}{%
\chapter{PhD Thesis}\label{phd-thesis}}

Thesis research is normally actively undertaken following successfully passing the comprehensive Exam and defending the thesis protocol. It is expected that students will complete their degree within 48 to 60 months of entry into the PhD degree program. The thesis research begins with

A thesis for the Doctoral degree must constitute original scholarship and must be a distinct contribution to knowledge. It must show familiarity with previous work in the field and must demonstrate your ability to plan and carry out research, organize results, and defend the approach and conclusions in a scholarly manner. The research presented must meet current standards of the discipline and clearly demonstrate how the research advances knowledge in the field. Finally, the thesis must be written in compliance with norms for academic and scholarly expression and for publication in the public domain (see section: withholding a thesis from circulation temporarily). The nature of academic research requires adherence to McGill's policies on research ethics and intellectual property, as described below.

\hypertarget{human-subjects-review}{%
\section{Human Subjects Review}\label{human-subjects-review}}

Research involving human participants, animal subjects, micro-organisms, living cells, other biohazards, and/or radioactive materials must have had the appropriate compliance certification. Copies of any certificates of compliance must be retained by the supervisor and student in accordance with McGill's policies on research ethics. Supervisors indicate on the Nomination of Examiners and Thesis Submission Form that the thesis research has complied with all ethical standards. See the Ethics and Compliance webpage for further information about certification and training requirements.

\hypertarget{section-6}{%
\section{---}\label{section-6}}

\hypertarget{policies}{%
\chapter{Policies}\label{policies}}

\hypertarget{progress-tracking}{%
\section{Progress Tracking}\label{progress-tracking}}

\hypertarget{research-ethics}{%
\section{Research Ethics}\label{research-ethics}}

\hypertarget{fast-track-from-msc-to-phd}{%
\section{Fast-Track from MSc to PhD}\label{fast-track-from-msc-to-phd}}

The transfer policy applies ONLY to students in Epidemiology and Biostatistics programs. A student who has been accepted into the M.Sc. program can request transfer into the Ph.D.~program. The formal transfer into the Ph.D.~program should occur within 12 months of initial enrollment in the Master's program.

In order to transfer from the M.Sc. to the Ph.D.~program, the M.Sc. student must complete all required courses for the M.Sc. program with a minimum GPA of 3.7.

REQUIRED DOCUMENTS
Biostatistics
A three-page outline of the proposed Ph.D.~thesis proposal;
Letters of support from the current M.Sc. supervisor(s) and proposed Ph.D.~supervisor(s). In cases where this is the same person, one additional letter from a faculty member in the Department is required. The letter from the proposed Ph.D.~supervisor(s) must include a statement of financial support for the candidate and research.
Epidemiology

A three to five-page outline of the proposed Ph.D.~thesis proposal including:
Research question
Background and brief literature review indicating the importance of the proposed research
Preliminary research design
Data source
Preliminary analysis plan
A timeline for completion of the Ph.D.~program
Letters of support from the current M.Sc. supervisor(s) and proposed Ph.D.~supervisor(s). In cases where this is the same person, one additional letter from a faculty member in the Department is required. The letter from the proposed Ph.D.~supervisor(s) must include a statement of financial support for the candidate and research.
The transfer application material must be submitted by April 25 to \href{mailto:gradadmin.eboh@mcgill.ca}{\nolinkurl{gradadmin.eboh@mcgill.ca}}.

\hypertarget{policy-on-departmental-seminars}{%
\section{Policy on Departmental Seminars}\label{policy-on-departmental-seminars}}

Updated November 9th, 2015
The Department considers attendance at its seminars an important component of training and expects all students to attend as frequently as possible.

All Epidemiology and Public Health graduate students will be expected to attend 60\% of the Epidemiology Seminars each term irrespective of their program year. Attendance at seminars will be mandatory to maintain ``good standing'' during academic studies. This means that attendance at seminars will be required to maintain eligibility for Departmental support for prizes, financial aid, travel awards, studentship applications, etc. Attendance at Biostatistics or at other seminars on campus or in teaching hospitals will be encouraged as always depending on one's areas of interest but will not be mandatory.

Students who will be unable to attend 60\% of the seminars should send a request for an exemption to the chairs office justifying their absence (eg. residing outside of Montreal, travel relating to their studies, family reasons).

\hypertarget{policy-on-email}{%
\section{Policy on Email}\label{policy-on-email}}

University Policy Concerning E-Mail As An Official Means Of Communication With Students. E-mail is an official means of communication between McGill University and its students.

In order to satisfy the need for timely and efficient communication, and to provide a better service to its students, McGill University has instituted a policy that establishes e-mail as an official means of communicating with students.

Upon registration at McGill, each student is assigned an official McGill e-mail address and a McGill e-mail box. This address may be viewed and verified via Minerva, under the Personal menu.

The McGill E-mail Address points to the McGill e-mail box by default for all students. As with all official University communications, it is the student's responsibility to ensure that time-critical e-mail is accessed, read, and acted upon in a timely fashion. If a student chooses to forward University e-mail to another e-mail mailbox, it is that student's responsibility to ensure that the alternate account is viable.

This policy applies to all McGill students and employees who manage official communications with students.

For confidential and official communication requiring an original signature, communication is by (physical) mail. Therefore, please ensure that your current postal address is updated on Minerva. Students must also inform the Student Affairs Office and complete the details in Minerva.

**Please keep in mind that although we do our best to keep all the information on the website up to date, we may have missed something. If you ever have questions about anything please contact the Student Affairs Office to clarify.

\hypertarget{section-7}{%
\section{---}\label{section-7}}

\hypertarget{funding}{%
\chapter{Funding}\label{funding}}

\hypertarget{deparmental-obligations}{%
\section{Deparmental Obligations}\label{deparmental-obligations}}

The Department, jointly with the PhD student's supervisor, offers an annual funding package valued at a \emph{minimum} of \$25,000 for a \emph{maximum} of four years. For international PhD students, the Department will use different mechanisms such as Deferential Fee Waivers, to cover the International Differential tuition so that they pay tuition at the same level as Canadian students.

Financial support is conditional upon academic performance, as well as \textbf{student efforts to apply for all external/internal graduate fellowship funding for which they are eligible}, including CIHR, FRQS, SSHRC, NSERC, etc., as well as the McGill Faculty of Medicine \& Health Sciences and Hospital Research Institute competitions.

\hypertarget{external-awards}{%
\section{External Awards}\label{external-awards}}

  \bibliography{book.bib}

\end{document}
